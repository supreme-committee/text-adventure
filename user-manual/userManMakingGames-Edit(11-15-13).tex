\documentclass[11pt]{article}
\usepackage{listings}
\begin{document}
\lstset{language=XML}
\title{How to Make Games Using FITR}
\author{Team Supreme Committee}
\date{November 2013}
\maketitle

\section{Introduction}

Our program is easy to understand. It simply provides a parser which reads .xml files of a specific format and displays them in a predetermined way. While we provide a sample game to show off the various features available, the real attraction is making your own games. As such we provide this simple guide to creating your own game.


\section{Organization}

\paragraph{Abstract View:}

In describing how to create games, it is useful to first break down how they are organized. Each game is made up of a compressed file (more details when we know more) which contains all the information necessary for each game. This primarily consists of .xml files, the building blocks of a game made for our engine. Each .xml file defines one 'tile' in the game, I will often refer to the collection of tiles as the 'map' even though this is slightly misleading since a tile does not necessarily represent a location in the game.

\paragraph{Tiles:}

When you play a game on our engine, what's on screen at any given time during the game is exactly one tile from the map. These tiles can be used in a variety of ways limited only by the developers ingenuity. They could for example, represent a physical location, a conversation, or a even one time events within the game. Each tile contains at least one 'link' to another tile, in this way developers can create large open worlds, scripted events, and allow players to make choices within the game. The syntax for creating these tiles is fairly simple, however debugging is difficult at best and it can be difficult to visualize how the tiles can interact so here is a sequence of 3 tiles which make up the beginning of our team's sample game.


\section{Example}

\paragraph{start.xml}

Here is the text of the start.xml tile which is the starting point for the sample game I have been creating with comments.

\begin{lstlisting}[frame=single]
<tile>
	<text>You are a wandering adventurer who after 
	weeks of traveling has made your way to 
	Fleurtran, the oldest of the great cities. You 
	stand in the town center, wondering what to do 
	next. There are two particularly busy streets 
	leading away from the plaza, one crowded with 
	throngs of people laden with goods and gold, 
	the other packed with soldiers hurrying up and 
	down it in squads. In front of you a large 
	message board catches your eye.</text>
	<image>magicforest.jpg</image>
	<link>
		<file>Market.xml</file>
		<text>Follow the crowds towards the 
		business district</text>
	</link>
	<link>
		<file>Barracks.xml</file>
		<text>Follow the soldiers towards the 
		barracks</text>
	</link>
	<link>
		<file>MessageBoard.xml</file>
		<text>Read the message board</text>
	</link>
	<link>
		<file>TownGate.xml</file>
		<text>Head towards the cities main 
		gate</text>
	</link>
</tile>
\end{lstlisting}
*Every 'map' must have a file named start.xml which is the default first tile loaded.

* Every tile must have the \textless tile\textgreater \textless /tile\textgreater tags marking the beginning and end.

* Each file must have some text which is displayed upon the player 'entering' this tile. This is marked with \textless text\textgreater yourtext\textless /text\textgreater . Simple, but also often contains the majority of the substance in most text games.

*The last of the 3 most basic and necessary tags, \textless link\textgreater yourlink\textless /link\textgreater  denotes a link to another tile. Each link must contain both a file (has the form \textless file\textgreater yourfile.ext\textless /file\textgreater and must be in the same directory as the file it is linked from) which is the tile it links to, and text to be displayed on the relevant button. It is worth noting that while it is not required to have a link in each tile, there is no 'go back' standard option, so a tile with no links should only be used in the case of winning or losing the game.

*In addition there is also the \textless image\textgreater yourimage.ext\textless /image\textgreater which sets the background to an image file whose path is specified relative to the .gamefiles directory (the directory created from the .tar file you give the game engine).

\paragraph{MessageBoard.xml}

\begin{lstlisting}[frame=single]
<tile>
	<text>There is only one message on the board, 
	its faded letters leaving you to believe it 
	has been posted for some time now: Action! 
	Adventure! Fair wages! Local merchant seeks 
	tough, independent, and self-motivated warrior 
	to guard their wares on a journey south across 
	the Objective Sea. Not for the faint of heart. 
	Inquire at the blue house just past the market 
	for details.</text>
	<var>
		<name>haveMerchantMessage</name>
		<value>true</value>
	</var> 
	<link>
		<file>TownSquare.xml</file>
		<text>Resume your inspection of the 
		town square</text>
	</link>
</tile>
\end{lstlisting}
*The main item of interest here is the variable tag. Each variable contains a name and a value. Both \textless name\textgreater yourname\textless /name\textgreater  and \textless value\textgreater yourvalue \textless /value\textgreater  can be set to any string. If a variable of the same name already exists this will overwrite its old value with the new one you provide.

*Straightforward uses of this include things like adding items to your inventory as in this example, recording which game events have already occurred allowing you to have one time scripted events, and assigning quantities to variables. The following file clarifies its use.

*Terrible programming jokes are necessary in any sample game.

\paragraph{Market.xml}

\begin{lstlisting}[frame=single]
<tile>
	<text>The cries of vendors mix with the smells 
	of baking food. A sea of people envelops you 
	as you make your way past countless tents 
	offering various services or products, often 
	brimming over with people united in their love 
	of goods and gold. Food and drink are the most 
	abundant, the city has seen better days, and 
	its venerability does not attract the luxury 
	goods and talent it once possessed. 
	Nevertheless, many fortunes have been made here 
	over the years, and beyond the chaos of the 
	market several rows of houses rise up, towering 
	over their neighbors, monuments to the power of 
	wealth. Tucked in between two such 
	monstrosities, a smaller blue house fights for 
	sunlight.</text>
	<if arg1="haveMerchantMessage" arg2="true" 
	comparison="eq">
        <link>
            <file>MerchantHouse.xml</file>
            <text>Message in hand, you make your way to 
		the blue house and knock on the 
		door.</text>
        </link>
	</if>
	<link>
		<file>TownSquare.xml</file>
		<text>Walk back to the town 
		square</text>
	</link>
	<link>
		<file>BuyFood.xml</file>
		<text>Delicious smells emanate from a 
		nearby stall. The plump vendor offers 
		you a loaf of bread and a block of 
		cheese in return for a small amount of 
		silver or a sharp bread knife</text>
	</link>
	<link>
		<file>BuyDrink.xml</file>
		<text>The man tending this stall is 
		stone faced and silent, in front of 
		him rows of water bottles glint in 
		the setting sun. In response to your 
		inquiry he merely points of a faded 
		sign above his head: Barter only, 
		make an offer</text>
	</link>
</tile>
\end{lstlisting}
*This .xml file demonstrates how one would use the variable declared in the tile before. It uses an if statement to check if a variable contains a specific string value. Usage is a little unusual so the next section is devoted to comparing variables.

*We can add values to objects by using an xml function that will allow us to assign gold values to objects and check with the amount of gold that the character has to see if they have enough money to buy the object.


\paragraph{Market.xml}

\begin{lstlisting}[frame=single]

<tile>
    <var>
        <name>fname</name>
        <value>Michael</value>
    </var>
    <var>
        <name>haverock</name>
        <value>true</value>
    </var>
    <var>
        <name>gold</name>
        <value>750</value>
    </var>
    <text>Hello {{fname}}. You have {{gold}} pieces of gold and haverock is {{haverock}}</text>
    <link>
        <file>start.xml</file>
        <text>Go</text>
    </link>
</tile>

It would look like this when displayed:

Hello Michael. You have 750 pieces of gold and haverock is true

\end{lstlisting}

\section{Comparisons}

If statements are a little tricky in our game due to the limitations of xml. The syntax takes the form: 

\textless if arg1="variable" arg2="literal" comparison="gt"\textgreater

\textless /if\textgreater

This if statement will check if arg1(variable) is greater than arg2(literal) and if so will execute the code preceding the closing if. There are three supported comparisons, "eq" (equals), "gt" (greater than), and "ge" (greater than or equal to). For bools and strings you can only use "eq" for comparison.

It is important to note that either arg1 or arg2 must be a variable, however it can be in either order, and you can compare a variable with a literal. It is also possible to compare a variable to another variable. However you cannot compare two literals.


\section{Game Design}

While simply understanding the various functions and their syntax described above is enough to begin creating your own game, I think it's worth briefly discussing how you can use these features to recreate common video game functionality. In this case you start inside a city, and must deal with the classic problem in video games, how do you teach the player how to play and introduce them to the plot without boring tutorials or walls of text at the beginning? First I use the variables to artificially restrict movement. You can travel to the town gate from the starting zone, but it remains closed until you own a weapon (represented as a variable). This forces the player to at least start the simple quest so they become familiar with the world and controls before they are released to the more open and dangerous world beyond the city walls. 

\section{Tips and Tricks}

Not every tile is a physical location, this is apparent in the example, the message board is geographically within the town square. If you select the BuyFood link it opens up a series of dialogue focused tiles with options for trading inventory items for food. Each potential offer requires its own special .xml file detailing the net result for the player if the transaction is completed. Unfortunately without comparative operators, we don't yet have the capability to have a quantity of a variable you can subtract from, instead such interactions must either involve barter or be carefully scripted so there can only be one possible result.

\end{document}